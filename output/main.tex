\documentclass[a4paper]{article}
\usepackage[cm, empty]{fullpage}
\usepackage{tikz, pgfplots}
\usetikzlibrary{plotmarks,external}

\pgfplotsset
{enlargelimits=0.03, xlabel near ticks, ylabel near ticks,
 yticklabel style={/pgf/number format/fixed, /pgf/number format/precision=5},
 minor tick num=9, scaled y ticks=false,
 legend pos=north west, legend cell align=left,
 legend style={font=\footnotesize\tt, row sep=-2pt}
}

\tikzexternalize

\begin{document}

\pgfplotsset
{scaled x ticks=base 10:-6,
 cycle list={{mark=x},
             {mark=o}, {mark=*, every mark/.append style={fill=blue!60!white}},
             {mark=*, every mark/.append style={fill=green!60!white}}, {mark=*, every mark/.append style={fill=red!60!white}},
             {mark=otimes}, {mark=otimes*, every mark/.append style={fill=blue!60!white}},
             {mark=otimes*, every mark/.append style={fill=green!60!white}}, {mark=otimes*, every mark/.append style={fill=red!60!white}},
             {mark=square}, {mark=square*, every mark/.append style={fill=blue!60!white}},
             {mark=square*, every mark/.append style={fill=green!60!white}}, {mark=square*, every mark/.append style={fill=red!60!white}},
             {mark=diamond}, {mark=diamond*, every mark/.append style={fill=blue!60!white}},
             {mark=diamond*, every mark/.append style={fill=green!60!white}}, {mark=diamond*, every mark/.append style={fill=red!60!white}}}
}

\begin{figure}[p]
 \centering
 \begin{tikzpicture}
  \begin{axis}
   [width=.5\textwidth, height=12cm, line width=0.1mm,
    only marks, mark options={black, mark size=1.5},
    xlabel=size of input, ylabel=time (sec.)]
   \input{ocaml1.time.out}
  \end{axis}
 \end{tikzpicture}%
 \begin{tikzpicture}
  \begin{axis}
   [width=.5\textwidth, height=12cm, line width=0.1mm,
    only marks, mark options={black, mark size=1.5},
    xlabel=size of input, ylabel=heap consumption (MB)]
   \input{ocaml1.mem.out}
  \end{axis}
 \end{tikzpicture}
 \caption{Benchmark result of \texttt{\textit{sort} xs} in OCaml, where \texttt{\textit{sort}} is \texttt{List.stable\_sort compare} or an extracted sorting function applied to \texttt{(<=)}, and \texttt{xs} is a list of random natural numbers of type \texttt{int}.}
\end{figure}

\begin{figure}[p]
 \centering
 \begin{tikzpicture}
  \begin{axis}
   [width=.5\textwidth, height=12cm, line width=0.1mm,
    only marks, mark options={black, mark size=1.5},
    xlabel=size of input, ylabel=time (sec.)]
   \input{ocaml2.time.out}
  \end{axis}
 \end{tikzpicture}%
 \begin{tikzpicture}
  \begin{axis}
   [width=.5\textwidth, height=12cm, line width=0.1mm,
    only marks, mark options={black, mark size=1.5},
    xlabel=size of input, ylabel=heap consumption (MB)]
   \input{ocaml2.mem.out}
  \end{axis}
 \end{tikzpicture}
 \caption{Benchmark result of \texttt{\textit{sort} xs} in OCaml, where \texttt{\textit{sort}} is \texttt{List.stable\_sort compare} or an extracted sorting function applied to \texttt{(<=)}, and \texttt{xs} is a list of random natural numbers of type \texttt{int} but its every block of length 50 is sorted in ascending order.}
\end{figure}

\pgfplotsset
{scaled x ticks=base 10:-6,
 cycle list={{mark=x},
             {mark=o}, {mark=*, every mark/.append style={fill=blue!60!white}},
             {mark=*, every mark/.append style={fill=green!60!white}}, {mark=*, every mark/.append style={fill=red!60!white}},
             {mark=otimes}, {mark=otimes*, every mark/.append style={fill=blue!60!white}},
             {mark=otimes*, every mark/.append style={fill=green!60!white}}, {mark=otimes*, every mark/.append style={fill=red!60!white}},
             {mark=square}, {mark=square*, every mark/.append style={fill=blue!60!white}},
             {mark=square*, every mark/.append style={fill=green!60!white}}, {mark=square*, every mark/.append style={fill=red!60!white}},
             {mark=diamond}, {mark=diamond*, every mark/.append style={fill=blue!60!white}},
             {mark=diamond*, every mark/.append style={fill=green!60!white}}, {mark=diamond*, every mark/.append style={fill=red!60!white}}},
}

\begin{figure}[p]
 \centering
 \begin{tikzpicture}
  \begin{axis}
   [width=\textwidth, height=12cm, line width=0.1mm,
    only marks, mark options={black, mark size=1.5},
    xlabel=size of input, ylabel=time (sec.)]
   \input{haskell1.time.out}
  \end{axis}
 \end{tikzpicture}
 \caption{Benchmark result of \texttt{sorted (\textcolor{red}{take 1000} (\textit{sort} xs))} in Haskell, where \texttt{\textit{sort}} is \texttt{Data.List.sort} or an extracted sorting function applied to \texttt{(<=)}, and \texttt{xs} is a list of random natural numbers of type \texttt{Int}.}
\end{figure}

\begin{figure}[p]
 \centering
 \begin{tikzpicture}
  \begin{axis}
   [width=\textwidth, height=12cm, line width=0.1mm,
    only marks, mark options={black, mark size=1.5},
    xlabel=size of input, ylabel=time (sec.)]
   \input{haskell2.time.out}
  \end{axis}
 \end{tikzpicture}
 \caption{Benchmark result of \texttt{sorted (\textcolor{red}{take 1000} (\textit{sort} xs))} in Haskell, where \texttt{\textit{sort}} is \texttt{Data.List.sort} or an extracted sorting function applied to \texttt{(<=)}, and \texttt{xs} is a list of random natural numbers of type \texttt{Int} but its every block of length 50 is sorted in ascending order.}
\end{figure}

\begin{figure}[p]
 \centering
 \begin{tikzpicture}
  \begin{axis}
   [width=\textwidth, height=12cm, line width=0.1mm,
    only marks, mark options={black, mark size=1.5},
    xlabel=size of input, ylabel=time (sec.)]
   \input{haskell3.time.out}
  \end{axis}
 \end{tikzpicture}
 \caption{Benchmark result of \texttt{sorted (\textit{sort} xs)} in Haskell, where \texttt{\textit{sort}} is \texttt{Data.List.sort} or an extracted sorting function applied to \texttt{(<=)}, and \texttt{xs} is a list of random natural numbers of type \texttt{Int}.}
\end{figure}

\begin{figure}[p]
 \centering
 \begin{tikzpicture}
  \begin{axis}
   [width=\textwidth, height=12cm, line width=0.1mm,
    only marks, mark options={black, mark size=1.5},
    xlabel=size of input, ylabel=time (sec.)]
   \input{haskell4.time.out}
  \end{axis}
 \end{tikzpicture}
 \caption{Benchmark result of \texttt{sorted (\textit{sort} xs)} in Haskell, where \texttt{\textit{sort}} is \texttt{Data.List.sort} or an extracted sorting function applied to \texttt{(<=)}, and \texttt{xs} is a list of random natural numbers of type \texttt{Int} but its every block of length 50 is sorted in ascending order.}
\end{figure}

\pgfplotsset
{scaled x ticks=base 10:-3,
 cycle list={{mark=o}, {mark=*, every mark/.append style={fill=blue!60!white}},
             {mark=*, every mark/.append style={fill=green!60!white}}, {mark=*, every mark/.append style={fill=red!60!white}},
             {mark=otimes}, {mark=otimes*, every mark/.append style={fill=blue!60!white}},
             {mark=otimes*, every mark/.append style={fill=green!60!white}}, {mark=otimes*, every mark/.append style={fill=red!60!white}},
             {mark=square}, {mark=square*, every mark/.append style={fill=blue!60!white}},
             {mark=square*, every mark/.append style={fill=green!60!white}}, {mark=square*, every mark/.append style={fill=red!60!white}},
             {mark=diamond}, {mark=diamond*, every mark/.append style={fill=blue!60!white}},
             {mark=diamond*, every mark/.append style={fill=green!60!white}}, {mark=diamond*, every mark/.append style={fill=red!60!white}}},
}

\begin{figure}[p]
 \centering
 \begin{tikzpicture}
  \begin{axis}
   [width=.5\textwidth, height=12cm, line width=0.1mm,
    only marks, mark options={black, mark size=1.5},
    xlabel=size of input, ylabel=time (sec.)]
   \input{lazy1.time.out}
  \end{axis}
 \end{tikzpicture}%
 \begin{tikzpicture}
  \begin{axis}
   [width=.5\textwidth, height=12cm, line width=0.1mm,
    only marks, mark options={black, mark size=1.5},
    xlabel=size of input, ylabel=heap consumption (MB)]
   \input{lazy1.mem.out}
  \end{axis}
 \end{tikzpicture}
 \caption{Benchmark result of \texttt{sorted N.leb (\textcolor{red}{take 10} (\textit{sort} N.leb xs))} with \texttt{lazy}, where \texttt{\textit{sort}} is a sorting function, and \texttt{xs} is a list of random natural numbers of type \texttt{N}.}
\end{figure}

\begin{figure}[p]
 \centering
 \begin{tikzpicture}
  \begin{axis}
   [width=.5\textwidth, height=12cm, line width=0.1mm,
    only marks, mark options={black, mark size=1.5},
    xlabel=size of input, ylabel=time (sec.)]
   \input{lazy2.time.out}
  \end{axis}
 \end{tikzpicture}%
 \begin{tikzpicture}
  \begin{axis}
   [width=.5\textwidth, height=12cm, line width=0.1mm,
    only marks, mark options={black, mark size=1.5},
    xlabel=size of input, ylabel=heap consumption (MB)]
   \input{lazy2.mem.out}
  \end{axis}
 \end{tikzpicture}
 \caption{Benchmark result of \texttt{sorted N.leb (\textcolor{red}{take 10} (\textit{sort} N.leb xs))} with \texttt{lazy}, where \texttt{\textit{sort}} is a sorting function, and \texttt{xs} is a list of random natural numbers of type \texttt{N} but its every block of length 50 is sorted in ascending order.}
\end{figure}

\begin{figure}[p]
 \centering
 \begin{tikzpicture}
  \begin{axis}
   [width=.5\textwidth, height=12cm, line width=0.1mm,
    only marks, mark options={black, mark size=1.5},
    xlabel=size of input, ylabel=time (sec.)]
   \input{lazy3.time.out}
  \end{axis}
 \end{tikzpicture}%
 \begin{tikzpicture}
  \begin{axis}
   [width=.5\textwidth, height=12cm, line width=0.1mm,
    only marks, mark options={black, mark size=1.5},
    xlabel=size of input, ylabel=heap consumption (MB)]
   \input{lazy3.mem.out}
  \end{axis}
 \end{tikzpicture}
 \caption{Benchmark result of \texttt{sorted N.leb (\textit{sort} N.leb xs)} with \texttt{lazy}, where \texttt{\textit{sort}} is a sorting function, and \texttt{xs} is a list of random natural numbers of type \texttt{N}.}
\end{figure}

\begin{figure}[p]
 \centering
 \begin{tikzpicture}
  \begin{axis}
   [width=.5\textwidth, height=12cm, line width=0.1mm,
    only marks, mark options={black, mark size=1.5},
    xlabel=size of input, ylabel=time (sec.)]
   \input{lazy4.time.out}
  \end{axis}
 \end{tikzpicture}%
 \begin{tikzpicture}
  \begin{axis}
   [width=.5\textwidth, height=12cm, line width=0.1mm,
    only marks, mark options={black, mark size=1.5},
    xlabel=size of input, ylabel=heap consumption (MB)]
   \input{lazy4.mem.out}
  \end{axis}
 \end{tikzpicture}
 \caption{Benchmark result of \texttt{sorted N.leb (\textit{sort} N.leb xs)} with \texttt{lazy}, where \texttt{\textit{sort}} is a sorting function, and \texttt{xs} is a list of random natural numbers of type \texttt{N} but its every block of length 50 is sorted in ascending order.}
\end{figure}

\begin{figure}[p]
 \centering
 \begin{tikzpicture}
  \begin{axis}
   [width=.5\textwidth, height=12cm, line width=0.1mm,
    only marks, mark options={black, mark size=1.5},
    xlabel=size of input, ylabel=time (sec.)]
   \input{compute1.time.out}
  \end{axis}
 \end{tikzpicture}%
 \begin{tikzpicture}
  \begin{axis}
   [width=.5\textwidth, height=12cm, line width=0.1mm,
    only marks, mark options={black, mark size=1.5},
    xlabel=size of input, ylabel=heap consumption (MB)]
   \input{compute1.mem.out}
  \end{axis}
 \end{tikzpicture}
 \caption{Benchmark result of \texttt{sorted N.leb (\textit{sort} N.leb xs)} with \texttt{compute}, where \texttt{\textit{sort}} is a sorting function, and \texttt{xs} is a list of random natural numbers of type \texttt{N}.}
\end{figure}

\begin{figure}[p]
 \centering
 \begin{tikzpicture}
  \begin{axis}
   [width=.5\textwidth, height=12cm, line width=0.1mm,
    only marks, mark options={black, mark size=1.5},
    xlabel=size of input, ylabel=time (sec.)]
   \input{compute2.time.out}
  \end{axis}
 \end{tikzpicture}%
 \begin{tikzpicture}
  \begin{axis}
   [width=.5\textwidth, height=12cm, line width=0.1mm,
    only marks, mark options={black, mark size=1.5},
    xlabel=size of input, ylabel=heap consumption (MB)]
   \input{compute2.mem.out}
  \end{axis}
 \end{tikzpicture}
 \caption{Benchmark result of \texttt{sorted N.leb (\textit{sort} N.leb xs)} with \texttt{compute}, where \texttt{\textit{sort}} is a sorting function, and \texttt{xs} is a list of random natural numbers of type \texttt{N} but its every block of length 50 is sorted in ascending order.}
\end{figure}

\begin{figure}[p]
 \centering
 \begin{tikzpicture}
  \begin{axis}
   [width=.5\textwidth, height=12cm, line width=0.1mm,
    only marks, mark options={black, mark size=1.5},
    xlabel=size of input, ylabel=time (sec.)]
   \input{vm1.time.out}
  \end{axis}
 \end{tikzpicture}%
 \begin{tikzpicture}
  \begin{axis}
   [width=.5\textwidth, height=12cm, line width=0.1mm,
    only marks, mark options={black, mark size=1.5},
    xlabel=size of input, ylabel=heap consumption (MB)]
   \input{vm1.mem.out}
  \end{axis}
 \end{tikzpicture}
 \caption{Benchmark result of \texttt{sorted N.leb (\textit{sort} N.leb xs)} with \texttt{vm\_compute}, where \texttt{\textit{sort}} is a sorting function, and \texttt{xs} is a list of random natural numbers of type \texttt{N}.}
\end{figure}

\begin{figure}[p]
 \centering
 \begin{tikzpicture}
  \begin{axis}
   [width=.5\textwidth, height=12cm, line width=0.1mm,
    only marks, mark options={black, mark size=1.5},
    xlabel=size of input, ylabel=time (sec.)]
   \input{vm2.time.out}
  \end{axis}
 \end{tikzpicture}%
 \begin{tikzpicture}
  \begin{axis}
   [width=.5\textwidth, height=12cm, line width=0.1mm,
    only marks, mark options={black, mark size=1.5},
    xlabel=size of input, ylabel=heap consumption (MB)]
   \input{vm2.mem.out}
  \end{axis}
 \end{tikzpicture}
 \caption{Benchmark result of \texttt{sorted N.leb (\textit{sort} N.leb xs)} with \texttt{vm\_compute}, where \texttt{\textit{sort}} is a sorting function, and \texttt{xs} is a list of random natural numbers of type \texttt{N} but its every block of length 50 is sorted in ascending order.}
\end{figure}

\begin{figure}[p]
 \centering
 \begin{tikzpicture}
  \begin{axis}
   [width=.5\textwidth, height=12cm, line width=0.1mm,
    only marks, mark options={black, mark size=1.5},
    xlabel=size of input, ylabel=time (sec.)]
   \input{native1.time.out}
  \end{axis}
 \end{tikzpicture}%
 \begin{tikzpicture}
  \begin{axis}
   [width=.5\textwidth, height=12cm, line width=0.1mm,
    only marks, mark options={black, mark size=1.5},
    xlabel=size of input, ylabel=heap consumption (MB)]
   \input{native1.mem.out}
  \end{axis}
 \end{tikzpicture}
 \caption{Benchmark result of \texttt{sorted N.leb (\textit{sort} N.leb xs)} with \texttt{native\_compute}, where \texttt{\textit{sort}} is a sorting function, and \texttt{xs} is a list of random natural numbers of type \texttt{N}.}
\end{figure}

\begin{figure}[p]
 \centering
 \begin{tikzpicture}
  \begin{axis}
   [width=.5\textwidth, height=12cm, line width=0.1mm,
    only marks, mark options={black, mark size=1.5},
    xlabel=size of input, ylabel=time (sec.)]
   \input{native2.time.out}
  \end{axis}
 \end{tikzpicture}%
 \begin{tikzpicture}
  \begin{axis}
   [width=.5\textwidth, height=12cm, line width=0.1mm,
    only marks, mark options={black, mark size=1.5},
    xlabel=size of input, ylabel=heap consumption (MB)]
   \input{native2.mem.out}
  \end{axis}
 \end{tikzpicture}
 \caption{Benchmark result of \texttt{sorted N.leb (\textit{sort} N.leb xs)} with \texttt{native\_compute}, where \texttt{\textit{sort}} is a sorting function, and \texttt{xs} is a list of random natural numbers of type \texttt{N} but its every block of length 50 is sorted in ascending order.}
\end{figure}

\end{document}
